\documentclass[]{article}
\usepackage[T1]{fontenc}
\usepackage{lmodern}
\usepackage{amssymb,amsmath}
\usepackage{ifxetex,ifluatex}
\usepackage{fixltx2e} % provides \textsubscript
% use upquote if available, for straight quotes in verbatim environments
\IfFileExists{upquote.sty}{\usepackage{upquote}}{}
\ifnum 0\ifxetex 1\fi\ifluatex 1\fi=0 % if pdftex
  \usepackage[utf8]{inputenc}
\else % if luatex or xelatex
  \ifxetex
    \usepackage{mathspec}
    \usepackage{xltxtra,xunicode}
  \else
    \usepackage{fontspec}
  \fi
  \defaultfontfeatures{Mapping=tex-text,Scale=MatchLowercase}
  \newcommand{\euro}{€}
\fi
% use microtype if available
\IfFileExists{microtype.sty}{\usepackage{microtype}}{}
\ifxetex
  \usepackage[setpagesize=false, % page size defined by xetex
              unicode=false, % unicode breaks when used with xetex
              xetex]{hyperref}
\else
  \usepackage[unicode=true]{hyperref}
\fi
\hypersetup{breaklinks=true,
            bookmarks=true,
            pdfauthor={David Grisham},
            pdftitle={Encoding Logic Gates in Ising Hamiltonian},
            colorlinks=true,
            citecolor=blue,
            urlcolor=blue,
            linkcolor=magenta,
            pdfborder={0 0 0}}
\urlstyle{same}  % don't use monospace font for urls
\setlength{\parindent}{0pt}
\setlength{\parskip}{6pt plus 2pt minus 1pt}
\setlength{\emergencystretch}{3em}  % prevent overfull lines
\setcounter{secnumdepth}{0}
\usepackage{braket} \newcommand{\su}{\ensuremath{\uparrow}}
\newcommand{\sd}{\ensuremath{\downarrow}}
\newcommand{\trans}{\ensuremath{\rightarrow}} \def\tightlist{}

\title{Encoding Logic Gates in Ising Hamiltonian}
\author{David Grisham}
\date{}

\begin{document}
\maketitle

\vspace{-0.8cm}

\section{\texorpdfstring{Determining Field (\(h_i\)) and Interaction
(\(J_{ij}\))Terms}{Determining Field (h\_i) and Interaction (J\_\{ij\})Terms}}\label{determining-field-hux5fi-and-interaction-jux5fijterms}

\subsection{General Outline of the
Steps}\label{general-outline-of-the-steps}

\begin{itemize}
\tightlist
\item
  Start with any spin configuration, then start flipping individual
  qubits and think about how the energy of the system should
  \emph{change} with those flips.
\item
  When doing this, keep in mind the following:

  \begin{enumerate}
  \def\labelenumi{\arabic{enumi}.}
  \tightlist
  \item
    The lowest energy states should correspond to the states of the
    logic-gate of interest. So, when switching from a state that encodes
    one input/output set for our gate to one that does not, the energy
    should increase. We can think of this in terms of work/energy from
    classical mechanics, with this case being analogous to nonzero work
    being done on our system, since the total energy of the system
    changes.
  \item
    The states that correspond to our logic gate all have the same
    energy. This puts a restriction on how certain values (terms in the
    total energy equation) have to change with bit flips (i.e.~do a bit
    flip, see which interaction terms and which field terms determine
    the change in energy of the system. If assumptions have been made
    about most of them, then the fact that the total energy should be
    the same can dictate the value of the final field/interaction terms
    we need). Following the work/energy analogy again, this would be the
    case where there is no work done in our system yet the relative
    values of internal energies change.
  \end{enumerate}
\end{itemize}

\section{Concrete Example}\label{concrete-example}

As an example, let's look at how certain bit flips change the energy of
our system in the NAND gate example that Prof.~Coffey provided, where A
and B are the inputs to our NAND gate and C is the output. Recall that
spin UP (denoted \su) corresponds to logical True, and spin DOWN (\sd)
corresponds to logical False. Assume the state of our system is written
as \(\ket{s_a s_b s_c}\), where \(s_a\), \(s_b\), and \(s_c\) correspond
to the spins of inputs A, B, and C, respectively. Also recall that the
Ising Hamiltonian takes the following form:

\[H_f = \sum_j h_j s_j + \sum_{i < j} J_{ij} s_i s_j\]

\subsection{Case 1: Non-NAND to NAND
Transition}\label{case-1-non-nand-to-nand-transition}

Consider how the system's energy changes in the following transition:

\[\ket{\sd\sd\sd} \trans \ket{\sd\sd\su}\]

Notice first that the starting state, \(\ket{\sd\sd\sd}\), does not
correspond to one of the NAND gate configurations. Also notice that they
state we end up in, \(\ket{\sd\sd\su}\), corresponds to A and B being
False and C being True, which agrees with the NAND gate logic. Because
of this, we know that the energy of our system should be lower in the
second state, since the states that agree with the NAND gate logic
should be the only ground-states of our system. We also need to pay
attention to the spin that flipped in this transition, which was
\(s_c\). This means that, when \(s_c\) changed, the energy of our system
decreased. There are two key possible reasons for this:

\begin{enumerate}
\def\labelenumi{\arabic{enumi}.}
\tightlist
\item
  \(s_c\) flipped, which also means that its alignment with the
  \(\vec{B}\)-field term \(h_c\) flipped as well. We know the energy
  decreased, so it's possible that it is now anti-aligned with the
  component of the \(\vec{B}\)-field that corresponds to this qubit.
  However, it is also possible that this term increases the total
  energy, since we currently do not have an alignment assigned to the
  \(\vec{B}\)-field and there are other terms we must consider that
  could have decreased the total energy as well. We can only say that
  \emph{magnitude} of the change caused by this term is
  \(2 \times h_c\).
\item
  Before the transition, \(s_c\) was aligned with \(s_a\) and \(s_b\);
  after the transition, it was not. So the interaction constants
  \(J_{ac}\) and \(J_{bc}\) are of interest here. We know that
  \(J_{ac}=J_{bc}\), because A and B are equivalent as far as C is
  concerned. Let's assume that the interaction terms \(J_{ac}\) and
  \(J_{bc}\) are positive. Then, looking at the interaction terms in the
  Ising Hamiltonian, we see that if two spins \(s_i\) and \(s_j\) are
  aligned and their interaction term \(J_{ij}\) is positive, then the
  energy of the system includes a \textbf{positive} \(J_{ij}\) term. If
  they are anti-aligned, the total energy includes a \textbf{negative}
  \(J_{ij}\) term. Given that C went from being aligned with A and B to
  being anti-aligned with them, the total energy of our system must have
  \emph{decreased} by \(2 \times (J_{ac} + J_{bc})\).
\end{enumerate}

Now we can write an equation that relates the non-zero changes in energy
within our system to the overall change (which, at this point, we only
know must be less than 0):

\[\pm 2 \times h_c - 2 \times (J_{ac} + J_{bc}) < 0\]

Notice that we still cannot say anything certain about the sign of the
field term, despite the assumptions made about the signs of the
interaction terms.

We made one assumption above, namely that the interaction terms
\(J_{ac}\) and \(J_{bc}\) were positive. This was not necesary, as there
are two possible terms that could decrease the energy of our system.
However, making assumptions such as this narrows down the form that our
problem will take; we could have said that those terms were negative
instead, which would change the arguments we could make in subsequent
parts of the analysis.

\subsection{Case 2: NAND to NAND
Transition}\label{case-2-nand-to-nand-transition}

Now let's consider a second transition, one that goes between two states
that agree with the NAND gate logic (and, thus, each correspond to a
ground state of our system):

\[\ket{\sd\sd\su} \trans \ket{\sd\su\su}\]

This time, we flipped \(s_b\). We know from the previous part that the
first of these states agrees with the NAND logic, and we can verify that
the second state, \(\ket{\sd\su\su}\), does as well (A is False, B is
True, so C should be True as well). As mentioned previously, that fact
that both states satisfy our gate logic means that they should both
correspond to the same ground state energy. We know that our field and
interaction constants are not simply 0, so there must have been a
tradeoff between certain energy values internal to our system.
Specifically, for the transition in question:

\begin{enumerate}
\def\labelenumi{\arabic{enumi}.}
\tightlist
\item
  \(s_b\) flipped, which means that its alignment with the relevant
  \(\vec{B}\)-field term, \(h_b\), must have flipped as well. As in the
  first case, we cannot tell without further analysis whether this term
  increased or decreased during the transition. The magnitude of the
  change caused by this term is \(2 \times h_b\).
\item
  Before the transition \(s_b\) was aligned with \(s_a\) only, and after
  the transition it was aligned with only \(s_c\). Our previous
  assumption that \(J_{bc}\) is positive tells us that the corresponding
  term in the total energy equation \emph{increased} by
  \(2 \times J_{bc}\). However, we still do not have enough information
  to tell whether \(J_{ab}\) is positive or negative. However, since we
  already assumed \(J_{ac}\) and \(J_{bc}\) were positive, we can simply
  argue for consistency and say that \(J_{ab}\) is also positive. This
  means that corresponding term in our total energy equation
  \emph{decreased} by \(2 \times  J_{ab}\), since \(s_a\) is spin DOWN
  and \(s_b\) went from spin DOWN to spin UP (which caused the sign
  flip).
\end{enumerate}

Given all of this information, we can now characterize the relationship
between the energy changes within our system as follows:

\[\pm 2 \times h_b + 2 \times (J_{bc} - J_{ab}) = 0\]

\section{Concluding Thoughts}\label{concluding-thoughts}

The process presented here is a rough outline of how we might go about
assigning signs and values to the field and interaction terms for a
Hamiltonian that represents a logic gate. There are steps that could be
taken, such as looking at additional transitions and the corresponding
changes in energy, to further simplify the analysis above. Once an
adequate number of constraints have been placed on the variables, one
could start assigning signs (as was done above) and concrete values to
the constants, since we really just care about the relative signs and
magnitudes of the \(h_i\) and \(J_{ij}\) values. Thus, there are
multiple sets of values that could be assigned to these constants, as
long as they obey all of the necessary constraints.

\end{document}
